
%%%%%%%%%%%%%%%%%%%%%%%%%%%%%%%%%%%%%%%%%%%%%%%%
%%% CV Guillermo González Martín 01/2026 %%%%%%
%%%%%%%%%%%%%%%%%%%%%%%%%%%%%%%%%%%%%%%%%%%%%%%%

\documentclass[10pt,a4paper,ragged2e,withhyper]{altacv}
%% AltaCV uses the fontawesome5 and simpleicons packages.
%% See http://texdoc.net/pkg/fontawesome5 and http://texdoc.net/pkg/simpleicons for full list of symbols.

% Change the page layout if you need to
\geometry{left=1cm,right=1cm,top=1cm,bottom=1cm,columnsep=0.8cm}

% The paracol package lets you typeset columns of text in parallel
\usepackage{paracol}

% Change the font if you want to, depending on whether
% you're using pdflatex or xelatex/lualatex
% WHEN COMPILING WITH XELATEX PLEASE USE
% xelatex -shell-escape -output-driver="xdvipdfmx -z 0" sample.tex
\iftutex
% If using xelatex or lualatex:
\setmainfont{Roboto Slab}
\setsansfont{Lato}
\renewcommand{\familydefault}{\sfdefault}

\else
% If using pdflatex:
  \usepackage[rm]{roboto}
  \usepackage[defaultsans]{lato}
  % \usepackage{sourcesanspro}
  \renewcommand{\familydefault}{\sfdefault}
\fi

% Change the colours if you want to

\definecolor{SlateGrey}{HTML}{2E2E2E}  % Color oscuro neutro → Mantener o usar un azul muy oscuro
\definecolor{LightGrey}{HTML}{666666}  % Color gris → Azul grisáceo
\definecolor{DarkPastelRed}{HTML}{001F33}  % Azul muy oscuro
\definecolor{PastelRed}{HTML}{005B8F}  % Azul medio más oscuro
\definecolor{GoldenEarth}{HTML}{7AA9D6}  % Azul claro ligeramente más oscuro
\colorlet{name}{black}
\colorlet{tagline}{PastelRed}  % Azul medio
\colorlet{heading}{DarkPastelRed}  % Azul muy oscuro
\colorlet{headingrule}{GoldenEarth}  % Azul claro
\colorlet{subheading}{PastelRed}  % Azul medio
\colorlet{accent}{PastelRed}  % Azul medio
\colorlet{emphasis}{SlateGrey}  % Mantener oscuro
\colorlet{body}{LightGrey}  % Azul grisáceo

% Change some fonts, if necessary
\renewcommand{\namefont}{\Huge\rmfamily\bfseries}
\renewcommand{\personalinfofont}{\footnotesize}
\renewcommand{\cvsectionfont}{\LARGE\rmfamily\bfseries}
\renewcommand{\cvsubsectionfont}{\large\bfseries}

% Change the bullets for itemize and rating marker
% for \cvskill if you want to
\renewcommand{\cvItemMarker}{{\small\textbullet}}
\renewcommand{\cvRatingMarker}{\faCircle}
% ...and the markers for the date/location for \cvevent
% \renewcommand{\cvDateMarker}{\faCalendar*[regular]}
% \renewcommand{\cvLocationMarker}{\faMapMarker*}

% If your CV/résumé is in a language other than English,
% then you probably want to change these so that when you
% copy-paste from the PDF or run pdftotext, the location
% and date marker icons for \cvevent will paste as correct
% translations. For example Spanish:
% \renewcommand{\locationname}{Ubicación}
% \renewcommand{\datename}{Fecha}

%% Use (and optionally edit if necessary) this .tex if you
%% want to use an author-year reference style like APA(6)
%% for your publication list
% \input{pubs-authoryear.cfg}

%% Use (and optionally edit if necessary) this .tex if you
%% want an originally numerical reference style like IEEE
%% for your publication list
% \input{pubs-num.cfg}

%% sample.bib contains your publications
% \addbibresource{sample.bib}

\begin{document}
\name{Guillermo González Martín}
\tagline{Embedded Software Engineer}
%% You can add multiple photos on the left or right
% \photoR{2.8cm}{Globe_High}
% \photoL{2.5cm}{Yacht_High,Suitcase_High}

\personalinfo{%
  % Not all of these are required!
  \email{guille.gonzalez.13@gmail.com}
  \phone{+34 610 24 82 57}
  % \mailaddress{Madrid, España}
  \location{Madrid, España}
  % \homepage{ggonzalezm.es}
  % \twitter{@twitterhandle}
  % \xtwitter{@x-handle}
  \linkedin{guillegonzzalez}
  \github{guillegonzzalez}
  % \orcid{0000-0000-0000-0000}
  %% You can add your own arbitrary detail with
  %% \printinfo{symbol}{detail}[optional hyperlink prefix]
  % \printinfo{\faPaw}{Hey ho!}[https://example.com/]

  %% Or you can declare your own field with
  %% \NewInfoFiled{fieldname}{symbol}[optional hyperlink prefix] and use it:
  % \NewInfoField{gitlab}{\faGitlab}[https://gitlab.com/]
  % \gitlab{your_id}
  %%
  %% For services and platforms like Mastodon where there isn't a
  %% straightforward relation between the user ID/nickname and the hyperlink,
  %% you can use \printinfo directly e.g.
  % \printinfo{\faMastodon}{@username@instace}[https://instance.url/@username]
  %% But if you absolutely want to create new dedicated info fields for
  %% such platforms, then use \NewInfoField* with a star:
  % \NewInfoField*{mastodon}{\faMastodon}
  %% then you can use \mastodon, with TWO arguments where the 2nd argument is
  %% the full hyperlink.
  % \mastodon{@username@instance}{https://instance.url/@username}
}

\makecvheader
%% Depending on your tastes, you may want to make fonts of itemize environments slightly smaller
% \AtBeginEnvironment{itemize}{\small}

%% Set the left/right column width ratio to 6:4.
\columnratio{0.55}

\cvsection{About me}

Embedded Software Engineer with solid experience in \textbf{C development} for embedded and \textbf{Linux-based systems}, working close to \textbf{electrical and electronic hardware}. Background in industrial IoT, real-time systems and low-level drivers, with experience in both startup environments and large, regulated companies. Strong interest in industrial systems, electronics and safety-critical applications.

% Start a 2-column paracol. Both the left and right columns will automatically
% break across pages if things get too long.
\begin{paracol}{2}

\cvsection{Experience}


\cvevent{Software Engineer}{Alten (Airbus project)}{May 2025 -- ongoing}{Madrid, Spain}
\begin{itemize}
  \item Embedded software development in C for safety-critical systems and Python scripting.
  \item Use of RTOS such as PikeOS and VxWorks.
  \item Involvement in design, development, QA and testing activities in a large industrial organization.
\end{itemize}

\divider

\cvevent{Hardware \& Firmware developer}{TycheTools}{Feb 2020 -- May 2025}{Madrid, Spain}
Company developing environmental sensors for \textbf{data centers}, cooling system control, and operation cost optimization.

\begin{itemize}
    \item \textbf{Hardware / Electronics:} Schematic capture, PCB design, HW reviews. PCB bring-up and testing. Integration of PCBs into mechanical designs.
    \item \textbf{Firmware / Drivers:} Firmware development in C for ARM-based microcontrollers. Development of low-level drivers and peripheral integration. Communication protocols (BLE Mesh, UART, SPI, I2C, etc.).
    \item \textbf{Linux / Systems:} Development of Linux-based systems using Yocto. Interaction between embedded firmware and Linux systems.
\end{itemize}

\divider

\cvevent{Electronic Research Internship}{GreenLSI (Department of Electronic Engineering, ETSIT)}{Sept 2019 -- Feb 2020}{Madrid, Spain}
\begin{itemize}
    \item Research and development of Bluetooth Mesh networks based on Nordic nRF52 microcontrollers.
\end{itemize}

\cvsection{Technical skills}

\cvtag{C/C++}
\cvtag{\simpleicon{python} Python}

\cvtag{ARM}
\cvtag{\simpleicon{nordicsemiconductor} Nordic}
\cvtag{\simpleicon{espressif} ESP}
\cvtag{AVR}
\cvtag{PIC}

\cvtag{\simpleicon{linux} Linux}
\cvtag{\simpleicon{yocto} Yocto}
\cvtag{\simpleicon{gnubash} Bash}
\cvtag{\simpleicon{git} Git}
\cvtag{\simpleicon{make} Makefile}

\cvtag{Serial comms (I2C, SPI, UART, RS, avionics protocols, etc.)}
\cvtag{WiFi}
\cvtag{BLE}
\cvtag{LoRa}
\cvtag{NFC}
\cvtag{MQTT}
\cvtag{Modbus}
\cvtag{SNMP}

\cvtag{KiCad}
\cvtag{API REST}
\cvtag{\simpleicon{mysql} SQL}
\cvtag{Web (Django, FastAPI)}
\cvtag{LVGL}
\cvtag{\simpleicon{jira} Jira (Agile)}

\switchcolumn

\cvsection{Education}

\cvevent{Master's in Telecommunications Engineering}{Universidad Politécica de Madrid}{2019 -- 2020}{Madrid, ES}
\textbf{Honors} in Master's Thesis: “Design and implementation of a mesh sensor network for monitoring electrical characteristics”

\divider

\cvevent{Bachelor's in Telecommunications Technologies and Services Engineering}{Universidad Politécica de Madrid}{2015 -- 2019}{Madrid, ES}

\divider

\cvevent{Scientific Baccalaureate}{Eurocolegio Casvi}{2013 -- 2015}{Villaviciosa, Madrid}

\cvsection{Languages}

\cvskill{Spanish}{5}

\divider

\cvskill{English}{4}

\medskip

\cvsection{Softskills}

\begin{itemize}
  \setlength{\itemindent}{0.5em}
  \item Teamwork
  \item Problem-solving
  \item Agile Methodologies, Scrum
  \item Observant
  \item Organized
\end{itemize}

\cvsection{Others}

\begin{itemize}
  \setlength{\itemindent}{0.5em}
  \item \textbf{Do It Yourself} projects, 3D printing, \textbf{self-hosted} services:
  \cvtag{\simpleicon{homeassistant} Home Assistant} \cvtag{\simpleicon{immich} Immich} \\ \cvtag{\simpleicon{influxdb} InfluxDB} \cvtag{\simpleicon{grafana} Grafana} \cvtag{Frigate}
  \item Strong interest in electronics, system troubleshooting and equipment repair.
  \item \textbf{Driving license B} and \textbf{own car}.
\end{itemize}
% \cvsection{Projects}

% \cvevent{Project 1}{Funding agency/institution}{}{}
% \begin{itemize}
% \item Details
% \end{itemize}

% \divider

% \cvevent{Project 2}{Funding agency/institution}{Project duration}{}
% A short abstract would also work.

% \medskip

% \cvsection{A Day of My Life}

% % Adapted from @Jake's answer from http://tex.stackexchange.com/a/82729/226
% % \wheelchart{outer radius}{inner radius}{
% % comma-separated list of value/text width/color/detail}
% \wheelchart{1.5cm}{0.5cm}{%
%   6/8em/accent!30/{Sleep,\\beautiful sleep},
%   3/8em/accent!40/Hopeful novelist by night,
%   8/8em/accent!60/Daytime job,
%   2/10em/accent/Sports and relaxation,
%   5/6em/accent!20/Spending time with family
% }

% use ONLY \newpage if you want to force a page break for
% ONLY the current column
% \newpage

% \cvsection{Publications}

%% Specify your last name(s) and first name(s) as given in the .bib to automatically bold your own name in the publications list.
%% One caveat: You need to write \bibnamedelima where there's a space in your name for this to work properly; or write \bibnamedelimi if you use initials in the .bib
%% You can specify multiple names, especially if you have changed your name or if you need to highlight multiple authors.
% \mynames{Lim/Lian\bibnamedelima Tze,
%   Wong/Lian\bibnamedelima Tze,
%   Lim/Tracy,
%   Lim/L.\bibnamedelimi T.}
%% MAKE SURE THERE IS NO SPACE AFTER THE FINAL NAME IN YOUR \mynames LIST

% \nocite{*}

% \printbibliography[heading=pubtype,title={\printinfo{\faBook}{Books}},type=book]

% \divider

% \printbibliography[heading=pubtype,title={\printinfo{\faFile*[regular]}{Journal Articles}},type=article]

% \divider

% \printbibliography[heading=pubtype,title={\printinfo{\faUsers}{Conference Proceedings}},type=inproceedings]

%% Switch to the right column. This will now automatically move to the second
%% page if the content is too long.
% \switchcolumn

% \cvsection{My Life Philosophy}

% \begin{quote}
% ``Something smart or heartfelt, preferably in one sentence.''
% \end{quote}

% \cvsection{Most Proud of}

% \cvachievement{\faTrophy}{Fantastic Achievement}{and some details about it}
% \divider
% \cvachievement{\faHeartbeat}{Another achievement}{more details about it of course}
% \divider
% \cvachievement{\faHeartbeat}{Another achievement}{more details about it of course}

% \cvsection{Strengths}

% % Don't overuse these \cvtag boxes — they're just eye-candies and not essential. If something doesn't fit on a single line, it probably works better as part of an itemized list (probably inlined itemized list), or just as a comma-separated list of strengths.

% \cvtag{Hard-working}
% \cvtag{Eye for detail}\\
% \cvtag{Motivator \& Leader}
% \divider\smallskip
% \cvtag{C++}
% \cvtag{Embedded Systems}\\
% \cvtag{Statistical Analysis}

%% Yeah I didn't spend too much time making all the
%% spacing consistent... sorry. Use \smallskip, \medskip,
%% \bigskip, \vspace etc to make adjustments.
% \medskip


% \cvsection{Referees}

% % \cvref{name}{email}{mailing address}
% \cvref{Prof.\ Alpha Beta}{Institute}{a.beta@university.edu}
% {Address Line 1\\Address line 2}

% \divider

% \cvref{Prof.\ Gamma Delta}{Institute}{g.delta@university.edu}
% {Address Line 1\\Address line 2}

\end{paracol}

\end{document}
