
%%%%%%%%%%%%%%%%%%%%%%%%%%%%%%%%%%%%%%%%%%%%%%%%
%%% CV Guillermo González Martín 03/2025 %%%%%%
%%%%%%%%%%%%%%%%%%%%%%%%%%%%%%%%%%%%%%%%%%%%%%%%

\documentclass[10pt,a4paper,ragged2e,withhyper]{altacv}
%% AltaCV uses the fontawesome5 and simpleicons packages.
%% See http://texdoc.net/pkg/fontawesome5 and http://texdoc.net/pkg/simpleicons for full list of symbols.

% Change the page layout if you need to
\geometry{left=1.25cm,right=1.25cm,top=1.5cm,bottom=1.5cm,columnsep=1.2cm}

% The paracol package lets you typeset columns of text in parallel
\usepackage{paracol}

% Change the font if you want to, depending on whether
% you're using pdflatex or xelatex/lualatex
% WHEN COMPILING WITH XELATEX PLEASE USE
% xelatex -shell-escape -output-driver="xdvipdfmx -z 0" sample.tex
\iftutex
  % If using xelatex or lualatex:
  \setmainfont{Roboto Slab}
  \setsansfont{Lato}
  \renewcommand{\familydefault}{\sfdefault}
\else
  % If using pdflatex:
  \usepackage[rm]{roboto}
  \usepackage[defaultsans]{lato}
  % \usepackage{sourcesanspro}
  \renewcommand{\familydefault}{\sfdefault}
\fi

% Change the colours if you want to

% RED
% \definecolor{SlateGrey}{HTML}{2E2E2E}
% \definecolor{LightGrey}{HTML}{666666}
% \definecolor{DarkPastelRed}{HTML}{450808}
% \definecolor{PastelRed}{HTML}{8F0D0D}
% \definecolor{GoldenEarth}{HTML}{E7D192}
% \colorlet{name}{black}
% \colorlet{tagline}{PastelRed}
% \colorlet{heading}{DarkPastelRed}
% \colorlet{headingrule}{GoldenEarth}
% \colorlet{subheading}{PastelRed}
% \colorlet{accent}{PastelRed}
% \colorlet{emphasis}{SlateGrey}
% \colorlet{body}{LightGrey}

% BLUE
% \definecolor{SlateGrey}{HTML}{2E2E2E}  % Color oscuro neutro → Mantener o usar un azul muy oscuro
% \definecolor{LightGrey}{HTML}{666666}  % Color gris → Azul grisáceo
% \definecolor{DarkPastelRed}{HTML}{003366}  % Azul oscuro
% \definecolor{PastelRed}{HTML}{0077B6}  % Azul medio vibrante
% \definecolor{GoldenEarth}{HTML}{89CFF0}  % Azul claro
% \colorlet{name}{black}
% \colorlet{tagline}{PastelRed}  % Azul medio
% \colorlet{heading}{DarkPastelRed}  % Azul oscuro
% \colorlet{headingrule}{GoldenEarth}  % Azul claro
% \colorlet{subheading}{PastelRed}  % Azul medio
% \colorlet{accent}{PastelRed}  % Azul medio
% \colorlet{emphasis}{SlateGrey}  % Mantener oscuro
% \colorlet{body}{LightGrey}  % Azul grisáceo

% DARKER BLUE
\definecolor{SlateGrey}{HTML}{2E2E2E}  % Color oscuro neutro → Mantener o usar un azul muy oscuro
\definecolor{LightGrey}{HTML}{666666}  % Color gris → Azul grisáceo
\definecolor{DarkPastelRed}{HTML}{001F33}  % Azul muy oscuro
\definecolor{PastelRed}{HTML}{005B8F}  % Azul medio más oscuro
\definecolor{GoldenEarth}{HTML}{7AA9D6}  % Azul claro ligeramente más oscuro
\colorlet{name}{black}
\colorlet{tagline}{PastelRed}  % Azul medio
\colorlet{heading}{DarkPastelRed}  % Azul muy oscuro
\colorlet{headingrule}{GoldenEarth}  % Azul claro
\colorlet{subheading}{PastelRed}  % Azul medio
\colorlet{accent}{PastelRed}  % Azul medio
\colorlet{emphasis}{SlateGrey}  % Mantener oscuro
\colorlet{body}{LightGrey}  % Azul grisáceo

% Change some fonts, if necessary
\renewcommand{\namefont}{\Huge\rmfamily\bfseries}
\renewcommand{\personalinfofont}{\footnotesize}
\renewcommand{\cvsectionfont}{\LARGE\rmfamily\bfseries}
\renewcommand{\cvsubsectionfont}{\large\bfseries}

% Change the bullets for itemize and rating marker
% for \cvskill if you want to
\renewcommand{\cvItemMarker}{{\small\textbullet}}
\renewcommand{\cvRatingMarker}{\faCircle}
% ...and the markers for the date/location for \cvevent
% \renewcommand{\cvDateMarker}{\faCalendar*[regular]}
% \renewcommand{\cvLocationMarker}{\faMapMarker*}

% If your CV/résumé is in a language other than English,
% then you probably want to change these so that when you
% copy-paste from the PDF or run pdftotext, the location
% and date marker icons for \cvevent will paste as correct
% translations. For example Spanish:
% \renewcommand{\locationname}{Ubicación}
% \renewcommand{\datename}{Fecha}

%% Use (and optionally edit if necessary) this .tex if you
%% want to use an author-year reference style like APA(6)
%% for your publication list
% \input{pubs-authoryear.cfg}

%% Use (and optionally edit if necessary) this .tex if you
%% want an originally numerical reference style like IEEE
%% for your publication list
% \input{pubs-num.cfg}

%% sample.bib contains your publications
% \addbibresource{sample.bib}

\begin{document}
\name{Guillermo González Martín}
\tagline{Ingeniero de hardware y firmware}
%% You can add multiple photos on the left or right
% \photoR{2.8cm}{Globe_High}
% \photoL{2.5cm}{Yacht_High,Suitcase_High}

\personalinfo{%
  % Not all of these are required!
  \email{guille.gonzalez.13@gmail.com}
  \phone{+34 610 24 82 57}
  % \mailaddress{Madrid, España}
  \location{Madrid, España}
  % \homepage{ggonzalezm.es}
  % \twitter{@twitterhandle}
  % \xtwitter{@x-handle}
  \linkedin{guillegonzzalez}
  \github{guillegonzzalez}
  % \orcid{0000-0000-0000-0000}
  %% You can add your own arbitrary detail with
  %% \printinfo{symbol}{detail}[optional hyperlink prefix]
  % \printinfo{\faPaw}{Hey ho!}[https://example.com/]

  %% Or you can declare your own field with
  %% \NewInfoFiled{fieldname}{symbol}[optional hyperlink prefix] and use it:
  % \NewInfoField{gitlab}{\faGitlab}[https://gitlab.com/]
  % \gitlab{your_id}
  %%
  %% For services and platforms like Mastodon where there isn't a
  %% straightforward relation between the user ID/nickname and the hyperlink,
  %% you can use \printinfo directly e.g.
  % \printinfo{\faMastodon}{@username@instace}[https://instance.url/@username]
  %% But if you absolutely want to create new dedicated info fields for
  %% such platforms, then use \NewInfoField* with a star:
  % \NewInfoField*{mastodon}{\faMastodon}
  %% then you can use \mastodon, with TWO arguments where the 2nd argument is
  %% the full hyperlink.
  % \mastodon{@username@instance}{https://instance.url/@username}
}

\makecvheader
%% Depending on your tastes, you may want to make fonts of itemize environments slightly smaller
% \AtBeginEnvironment{itemize}{\small}

%% Set the left/right column width ratio to 6:4.
\columnratio{0.55}

\cvsection{Sobre mí}

Ingeniero de Telecomunicaciones con gran interés por los \textbf{sistemas electrónicos}, los \textbf{sistemas de potencia}, el IoT y la programación. Especializado en \textbf{sistemas empotrados} y en sistemas basados en \textbf{Linux}. Me encanta enfrentar nuevos desafíos y trabajar en un entorno innovador. Comprometido con el aprendizaje continuo, me adapto rápidamente a nuevas tecnologías y metodologías y preparado para desarrollar nuevas habilidades.

% Start a 2-column paracol. Both the left and right columns will automatically
% break across pages if things get too long.
\begin{paracol}{2}


\cvsection{Experiencia}

\cvevent{Desarrollador Hardware y Firmware}{TycheTools}{Febrero 2020 -- actualidad}{Madrid, ES}
Empresa desarrolladora de sensores ambientales para \textbf{centros de datos}, control de sistemas de refrigeración y optimización de costes de operación.

\begin{itemize}
\item \textbf{Desarrollo de hardware} analógico y digital (toma de requisitos, diseño, desarrollo y puesta en mercado de varios dispositivos). Captura de esquemáticos, diseño de PCB, revisión de HW. Integración de PCBs en diseños mecánicos.
\item \textbf{Ensamblado de PCBs:} soldadura manual TH, SMD (0402), testing HW.
\item \textbf{Compra de componentes} (Mouser, RS, Farnell, …).
\item \textbf{Gestión de compras}, \textbf{comunicación con proveedores}, \textbf{RFQs}.
\item \textbf{Desarrollo de firmware} para sistemas empotrados (principalmente para nRF52).
\item Comunicaciones \textbf{Bluetooth 5.0 Mesh}.
\item Desarrollo de sistema \textbf{Linux basado en Yocto}.
\item Experiencia en \textbf{centros de datos}: optimización de refrigeración y consumo energético.
\end{itemize}

\divider

\cvevent{Prácticas de investigación electrónica}{GreenLSI (Departamento de Ingeniería Electrónica, ETSIT)}{Septiembre 2019 -- Febrero 2020}{Madrid, ES}
\begin{itemize}
\item Investigación y desarrollo de redes Bluetooth Mesh Mesh basados en microcontroladores Nordic nRF52.
\end{itemize}


\cvsection{Capacidades técnicas}

\begin{itemize}
    \setlength{\itemindent}{0.5em}
    \item Lenguajes de programación: C/C++, Python, JS, Dart
    \item Microcontroladores: Nordic, ESP, AVR, PIC, ST
    \item Protocolos: I2C, SPI, UART, WiFi, BLE, MQTT, Modbus, SNMP, NFC, LoRa, SSH, RS232, RS485
    \item Gestión: Jira, Dolibarr, Holded
    \item Git (Github/Gitlab/Bitbucket), Bash, Linux, Docker, Yocto, LaTeX, Markdown, PlatformIO, Makefile
    \item KiCad, Autodesk Fusion 360
    \item VSCode, CAD, Inkscape, Blender
    \item API REST, NGINX, desarrollo web, SQL, Django, FastAPI, LVGL, Flutter, Raspberrys
    \item Sistemas de medición de potencia trifásica
    \item Desarrollo de producto, diseño 3D, diseño gráfico
\end{itemize}

\switchcolumn


\cvsection{Educación}

\cvevent{Máster en Ingeniería de Telecomunicaciones}{Universidad Politécnica de Madrid}{2019 --  2020}{Madrid, ES}
\textbf{Matrícula de Honor} en Trabajo Fin de Master: “Diseño e implementación de una red mallada de sensores para la monitorización de características eléctricas”

\divider

\cvevent{Grado en Ingeniería de Tecnologías y Servicios de Telecomunicaciones}{Universidad Politécnica de Madrid}{2015 -- 2019}{Madrid, ES}

\divider

\cvevent{Bachillerato científico}{Eurocolegio Casvi}{2013 -- 2015}{Villaviciosa, Madrid}

\cvsection{Idiomas}

\cvskill{Español}{5}
\divider

\cvskill{Inglés}{4}
\medskip

\cvsection{Softskills}

\begin{itemize}
    \setlength{\itemindent}{0.5em}
    \item Trabajo en equipo
    \item Solucionador de problemas
    \item Metodilogías Agile, Scrum
    \item Observador
    \item Organizado
\end{itemize}

\cvsection{Otros}

\begin{itemize}
    \setlength{\itemindent}{0.5em}
    \item Proyectos \textbf{"Do It Yourself”}, Impresión 3D, servicios \textbf{self-hosted} (Home Assistant, Immich, Frigate, Mosquitto, InfluxDB, Grafana).
    \item Electrónica, reparación de equipos.
    \item Permiso de \textbf{conducción B}.
\end{itemize}


% \cvsection{Projects}

% \cvevent{Project 1}{Funding agency/institution}{}{}
% \begin{itemize}
% \item Details
% \end{itemize}

% \divider

% \cvevent{Project 2}{Funding agency/institution}{Project duration}{}
% A short abstract would also work.

% \medskip

% \cvsection{A Day of My Life}

% % Adapted from @Jake's answer from http://tex.stackexchange.com/a/82729/226
% % \wheelchart{outer radius}{inner radius}{
% % comma-separated list of value/text width/color/detail}
% \wheelchart{1.5cm}{0.5cm}{%
%   6/8em/accent!30/{Sleep,\\beautiful sleep},
%   3/8em/accent!40/Hopeful novelist by night,
%   8/8em/accent!60/Daytime job,
%   2/10em/accent/Sports and relaxation,
%   5/6em/accent!20/Spending time with family
% }

% use ONLY \newpage if you want to force a page break for
% ONLY the current column
% \newpage

% \cvsection{Publications}

%% Specify your last name(s) and first name(s) as given in the .bib to automatically bold your own name in the publications list.
%% One caveat: You need to write \bibnamedelima where there's a space in your name for this to work properly; or write \bibnamedelimi if you use initials in the .bib
%% You can specify multiple names, especially if you have changed your name or if you need to highlight multiple authors.
% \mynames{Lim/Lian\bibnamedelima Tze,
%   Wong/Lian\bibnamedelima Tze,
%   Lim/Tracy,
%   Lim/L.\bibnamedelimi T.}
%% MAKE SURE THERE IS NO SPACE AFTER THE FINAL NAME IN YOUR \mynames LIST

% \nocite{*}

% \printbibliography[heading=pubtype,title={\printinfo{\faBook}{Books}},type=book]

% \divider

% \printbibliography[heading=pubtype,title={\printinfo{\faFile*[regular]}{Journal Articles}},type=article]

% \divider

% \printbibliography[heading=pubtype,title={\printinfo{\faUsers}{Conference Proceedings}},type=inproceedings]

%% Switch to the right column. This will now automatically move to the second
%% page if the content is too long.
% \switchcolumn

% \cvsection{My Life Philosophy}

% \begin{quote}
% ``Something smart or heartfelt, preferably in one sentence.''
% \end{quote}

% \cvsection{Most Proud of}

% \cvachievement{\faTrophy}{Fantastic Achievement}{and some details about it}

% \divider

% \cvachievement{\faHeartbeat}{Another achievement}{more details about it of course}

% \divider

% \cvachievement{\faHeartbeat}{Another achievement}{more details about it of course}

% \cvsection{Strengths}

% % Don't overuse these \cvtag boxes — they're just eye-candies and not essential. If something doesn't fit on a single line, it probably works better as part of an itemized list (probably inlined itemized list), or just as a comma-separated list of strengths.

% \cvtag{Hard-working}
% \cvtag{Eye for detail}\\
% \cvtag{Motivator \& Leader}

% \divider\smallskip

% \cvtag{C++}
% \cvtag{Embedded Systems}\\
% \cvtag{Statistical Analysis}


%% Yeah I didn't spend too much time making all the
%% spacing consistent... sorry. Use \smallskip, \medskip,
%% \bigskip, \vspace etc to make adjustments.
% \medskip


% \cvsection{Referees}

% % \cvref{name}{email}{mailing address}
% \cvref{Prof.\ Alpha Beta}{Institute}{a.beta@university.edu}
% {Address Line 1\\Address line 2}

% \divider

% \cvref{Prof.\ Gamma Delta}{Institute}{g.delta@university.edu}
% {Address Line 1\\Address line 2}


\end{paracol}


\end{document}
